\documentclass[a4paper,11pt,dvipdfmx]{ujarticle}

\usepackage{graphicx}
\usepackage{url}
\input{layout}

\title{インターネットの利用状況}

\author{G58400-2025 拓殖 太郎}

\begin{document}

\maketitle

\section{情報通信機器の保有状況}

総務省が毎年実施している通信利用動向調査\cite{soumu}によると,
図\ref{fig:保有率}に示すように,
情報通信機器の世帯保有率については,
携帯電話やスマートフォンなどのモバイル端末では,9割を越えている.
その中でも,スマートフォンの普及が進んでおり,8割以上の世帯で保有している.

\begin{figure}[htbp]
    \centering
    \includegraphics[width=0.7\linewidth]{possession.png}
    \caption{情報通信機器の世帯保有率}\label{fig:保有率}
\end{figure}

\section{端末の利用状況}

普段,私的な用途のために利用している端末で最も多いのは,
表\ref{tbl:利用状況}に示すように,
スマートフォン(89.4\%)で全体の9割近くが利用していた.
続いて,テレビ(50.8\%),ノートPC(48.5\%),タブレット(26.5\%)
の順で多く,テレビを除くと,持ち運びできる端末の利用が多い\cite{corona}.

\begin{table}[htbp]
    \centering
    \caption{モバイルブロードバンの加入者数}
    \label{tbl:利用状況}

    \begin{tabular}{|c|c|c|}
        \hline
        順位 & 国名 & 加入者数 \\
        \hline
        1位 & 日本 & 190.5\\
        \hline
        2位 & エストニア & 179.9\\
        \hline
        3位 & 米国 & 169.0\\
        \hline
        4位 & フィンランド & 157.0\\
        \hline
        5位 & デンマーク & 141.7\\
        \hline
        6位 & ラトビア & 141.6\\
        \hline 
        7位 & イスラエル & 139.9\\
        \hline
        8位 & オランダ & 133.7\\
        \hline
        9位 & ポーランド & 131.3\\
        \hline
        10位 & スウェーデン & 127.\\
        \hline
    \end{tabular}
\end{table}
